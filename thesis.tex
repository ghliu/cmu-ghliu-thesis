%for a more compact document, add the option openany to avoid
%starting all chapters on odd numbered pages
\documentclass[hidelinks, 12pt]{cmuthesis}

% This is a template for a CMU thesis.  It is 18 pages without any content :-)
% The source for this is pulled from a variety of sources and people.
% Here's a partial list of people who may or may have not contributed:
%
%        bnoble   = Brian Noble
%        caruana  = Rich Caruana
%        colohan  = Chris Colohan
%        jab      = Justin Boyan
%        josullvn = Joseph O'Sullivan
%        jrs      = Jonathan Shewchuk
%        kosak    = Corey Kosak
%        mjz      = Matt Zekauskas (mattz@cs)
%        pdinda   = Peter Dinda
%        pfr      = Patrick Riley
%        dkoes = David Koes (me)

% My main contribution is putting everything into a single class files and small
% template since I prefer this to some complicated sprawling directory tree with
% makefiles.

% some useful packages
\usepackage{times}
\usepackage{fullpage}
\usepackage{graphicx}
\usepackage{amsmath}
\usepackage{subfiles}
\usepackage[numbers,sort]{natbib}
\usepackage[backref,pageanchor=true,plainpages=false, pdfpagelabels, bookmarks,bookmarksnumbered,
%pdfborder=0 0 0,  %removes outlines around hyper links in online display
]{hyperref}
%% \usepackage{subfigure}

\usepackage{amsmath}
\usepackage{amssymb}
\usepackage{gensymb} 
\usepackage{dsfont}
\usepackage{graphicx}
\usepackage{float}
\let\labelindent\relax
\usepackage{enumitem}
\usepackage{hyperref}
\usepackage{tabu}
\usepackage{array}
\usepackage{caption}
\usepackage{subcaption}
\usepackage{colortbl}
\usepackage{tikz}
\usepackage{algorithm}
% \usepackage{algorithmic}
\usepackage{algpseudocode}
\usepackage{xparse}
\usetikzlibrary{positioning}


% \usepackage[utf8]{inputenc} % allow utf-8 input
\usepackage[T1]{fontenc}    % use 8-bit T1 fonts
% \usepackage{url}            % simple URL typesetting
\usepackage{booktabs}       % professional-quality tables
% \usepackage{amsfonts}       % blackboard math symbols
% \usepackage{nicefrac}       % compact symbols for 1/2, etc.
% \usepackage{microtype}      % microtypography
% \usepackage{}
\usepackage{multirow}
% \usepackage{pbox}
% \usepackage{enumitem}
% \usepackage{bbm}
% \usepackage{unicode-math}
% % \usepackage[table,xcdraw]{xcolor}
% \usepackage{xfrac}
% \usepackage{pifont}% http://ctan.org/pkg/pifont
% \usepackage{dblfloatfix}

% % For figures
% \graphicspath{{figs/}}

\usepackage[prependcaption,textsize=tiny]{todonotes}



\DeclareCaptionLabelFormat{andtable}{#1~#2  \&  \tablename~\thetable}
\DeclareDocumentCommand{\particles}{ O{} O{} O{}}{^{#2}X_{#1}^{#3}}
% \particle[time][index][section]
\DeclareDocumentCommand{\particle}{ O{} O{} O{}}{^{#2}x_{#1}^{#3}}
\newcommand{\state}{x}

\newcommand{\sampled}{\tilde}
\newcommand{\subcap}[1]{\textbf{(\subref{#1})}}
\newcommand{\maction}{\mathcal{M}}
%% \newcommand{\particles}{\mathcal{P}}
%% \newcommand{\particle}{x}
\newcommand{\bin}{b}
\newcommand{\groups}{L}
\newcommand{\measurement}{m}
\newcommand{\measurementSet}{M}
\newcommand{\totalWeight}{\mathcal{W}}
\newcommand{\feature}{\mathcal{S}}
\renewcommand{\algorithmicrequire}{\textbf{Input:}}
\renewcommand{\algorithmicensure}{\textbf{Output:}}
\def\BState{\State\hskip-\ALG@thistlm}

\algnewcommand\algorithmicforeach{\textbf{for each}}
\algdef{S}[FOR]{ForEach}[1]{\algorithmicforeach\ #1\ \algorithmicdo}


% Approximately 1" margins, more space on binding side
%\usepackage[letterpaper,twoside,vscale=.8,hscale=.75,nomarginpar]{geometry}
%for general printing (not binding)
\usepackage[letterpaper,twoside,vscale=.8,hscale=.75,nomarginpar,hmarginratio=1:1]{geometry}

% Provides a draft mark at the top of the document. 
% \draftstamp{\today}{DRAFT}

\begin {document} 
\frontmatter

%initialize page style, so contents come out right (see bot) -mjz
\pagestyle{empty}

\title{ %% {\it \huge Thesis Proposal}\\ \\ \protect
{\bf Planning and Learning for Off-Road Autonomous Navigation in High Dimensional State Space}}
\author{Guan-Horng Liu}
\trnumber{CMU-RI-TR-17-37}
\date{June 2017}
\Year{2017}

\committee{
George Kantor, Chair \\
Manuela Veloso \\
Devin Schwab 
}

\support{}
\disclaimer{}

% copyright notice generated automatically from Year and author.
% permission added if \permission{} given.

% \keywords{deep reinforcement learning, deep inverse reinforcement learning}

\maketitle

% \begin{dedication}
% For my family and Bonnie Chen
% \end{dedication}

\pagestyle{plain}

\begin{abstract}

% Recently, great strides have been made in the development of autonomous driving technologies, energized by the successful demonstrations by some teams at the DARPA Urban Challenge.
% While there is a foreseeable trend on operating autonomous driving technologies on-road within a decade, researchers have gained interests in solving extreme/aggressive motion planning in the off-road situations or unstructured terrain. 
% However, the high dimensional configuration space induced inevitably by a more complex vehicle model may break down traditional search-based algorithms with computational speed. 
% Alternatively, though end-to-end learning approaches have a great potential to handle high-dimensional state space, in the multi-sensor setting the learned policy may either rely heavily on all the inputs to the extent that it fails completely if a single sensor becomes unreliable, or become over-dependent on partial sensor subset, rendering sensor redundancy useless. 
% To be clear, in the space of end-to-end sensorimotor control, the sensor fusion outlook, as an indispensable technique to improve accuracy and robustness in a modern autonomous navigation system, has received limited attention. 

% it provides no guarantees on the sensitivity of the trained policy on each sensor input.
% the traditional search-based approach may break down once the motion planning shifts to high dimensional space, 
% it operates more like a black box function approximator, and crucial properties such as sensitivity

This thesis explores both traditional motion planning and end-to-end learning algorithms in the off-road settings.
We summarize the main contributions as 1) propose an RRT-based local planner for high-speed maneuvering, 2) derive a novel stochastic regularization technique that robustifies end-to-end learning in the spirit of sensor fusion, and 3) traversability analysis of the unstructured terrain using deep inverse reinforcement learning (DIRL) algorithms. 
We first propose a sample-based local planner that is modified to solve a minimal traveling-time trajectory problem subjected to a data-driven vehicle model in high dimensional state space.
The planner is implemented on a full-size All-Terrain Vehicle (ATV), and the experimental results show that it can successfully avoid obstacles on a turnpike with the vehicle velocity up to the maximum operating speed.
Secondly, we also propose a stochastic regularization technique, called \textit{Sensor} \textit{Dropout}, that promotes an effective fusing of information for end-to-end multimodal sensor policies. 
Through empirical testings on a physical-based racing car simulator called \textit{TORCS}, we demonstrate that our proposed policies can operate with minimal performance drop in noisy environments, and remain functional even in the face of a sensor subset failure. 
% Moreover, the policy network can automatically infer the weight locations that provide salient information. 
Lastly, we investigate into the DIRL algorithms that infer the traversability of the unstructured terrain by leveraging a large volume of human demonstration data collected on the field. 
We propose several modifications to overcome issues that arise from DIRL training, such as sparse gradients and ambiguity of the demonstration optimality.
The framework is tested on the full-size ATV. 
% over the current approach \cite{wulfmeier2015maximum} 
% , including explicitly modeling the ambiguity of optimality, and integrating with failure demonstrations to overcome the spatially sparse gradient in DIRL training.
% We show that the proposed planner can successfully avoid obstacles on a turnpike with vehicle velocity up to the maximum operating speed.


\end{abstract}

\begin{acknowledgments}

First, I would like to thank my advisor, Mr. George Kantor, for providing me advice, financial supports, and most importantly introducing me to the Yamaha ATV project that provides me a great opportunity to implement and test algorithms on the full-size autonomous vehicle.

I would also like to thank all of the members in the Yamaha ATV project, especially Jay Hiramatsu who supports most of the field experiments I need to finish this thesis. I appreciate every valuable discussion with many intelligent students here: Po-Wei, Wei-Hsin, Daniel, Sai. Discussions with senior Ph.D. students: Avinash Siravuru, Hsiao Ming, Humphrey Hu, and Devin Schwab inspire me frequently. 
I have big special thanks to Po-Wei Chou, both my project-mate and roommate, with whom I enjoy sharing my point of views and discuss. 
I would not make through my first year here without your help and collaborations.
% You help me catch up many most of the crucial knowledge especially in the first year of my master program.
% , that greatly release my pressure of transferring research field from a more hardware mechanical engineer to software.

Lastly, I would like to thank my parents for providing me sufficient resources and accesses to education in my life, and supporting most of my career decisions. I believe I owe so much to Bonnie Chen, for supporting me over the two years of long distance during the master program. I would not make all those tough times through without you beside me.

  
\end{acknowledgments}



\tableofcontents
\listoffigures
\listoftables

\mainmatter

%% Double space document for easy review:
%\renewcommand{\baselinestretch}{1.66}\normalsize

% The other requirements Catherine has:
%
%  - avoid large margins.  She wants the thesis to use fewer pages, 
%    especially if it requires colour printing.
%
%  - The thesis should be formatted for double-sided printing.  This
%    means that all chapters, acknowledgements, table of contents, etc.
%    should start on odd numbered (right facing) pages.
%
%  - You need to use the department standard tech report title page.  I
%    have tried to ensure that the title page here conforms to this
%    standard.
%
%  - Use a nice serif font, such as Times Roman.  Sans serif looks bad.
%
% Other than that, just make it look good...


\chapter{Introduction}
\subfile{Introduction/intro.tex}

\chapter{Related Work} \label{chap:related_work}
\subfile{RelatedWork/relatedWork.tex}

\chapter{Model Predictive Planning} \label{chap:rrtplanner}
\subfile{RRTPlanner/rrtplanner.tex}

\chapter{Learning End-to-end Multimodal Sensor Policy} \label{chap:multimodalDRL}
\subfile{MultimodalDRL/multimodalDRL.tex}

\chapter{Learning from Demonstration using DIRL} \label{chap:dirl}
\subfile{DIRL/dirl.tex}

\chapter{Conclusion and Future Work} \label{chap:conclusion}
\subfile{Conclusion/conclusion.tex}

%\appendix
%\include{appendix}

\backmatter

%\renewcommand{\baselinestretch}{1.0}\normalsize

% By default \bibsection is \chapter*, but we really want this to show
% up in the table of contents and pdf bookmarks.
\renewcommand{\bibsection}{\chapter{\bibname}}
%\newcommand{\bibpreamble}{This text goes between the ``Bibliography''
%  header and the actual list of references}
% \bibliographystyle{srtnat}
\bibliographystyle{unsrtnat}
\bibliography{MDRLBib,RRTPlannerBib,DIRLBib} %your bib file
\nocite{*}

\end{document}
