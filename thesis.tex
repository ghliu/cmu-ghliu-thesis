%for a more compact document, add the option openany to avoid
%starting all chapters on odd numbered pages
\documentclass[hidelinks, 12pt]{cmuthesis}

% This is a template for a CMU thesis.  It is 18 pages without any content :-)
% The source for this is pulled from a variety of sources and people.
% Here's a partial list of people who may or may have not contributed:
%
%        bnoble   = Brian Noble
%        caruana  = Rich Caruana
%        colohan  = Chris Colohan
%        jab      = Justin Boyan
%        josullvn = Joseph O'Sullivan
%        jrs      = Jonathan Shewchuk
%        kosak    = Corey Kosak
%        mjz      = Matt Zekauskas (mattz@cs)
%        pdinda   = Peter Dinda
%        pfr      = Patrick Riley
%        dkoes = David Koes (me)

% My main contribution is putting everything into a single class files and small
% template since I prefer this to some complicated sprawling directory tree with
% makefiles.

% some useful packages
\usepackage{times}
\usepackage{fullpage}
\usepackage{graphicx}
\usepackage{amsmath}
\usepackage{subfiles}
\usepackage[numbers,sort]{natbib}
\usepackage[backref,pageanchor=true,plainpages=false, pdfpagelabels, bookmarks,bookmarksnumbered,
%pdfborder=0 0 0,  %removes outlines around hyper links in online display
]{hyperref}
%% \usepackage{subfigure}

\usepackage{amsmath}
\usepackage{amssymb}
\usepackage{gensymb} 
\usepackage{dsfont}
\usepackage{graphicx}
\usepackage{float}
\let\labelindent\relax
\usepackage{enumitem}
\usepackage{hyperref}
\usepackage{tabu}
\usepackage{array}
\usepackage{caption}
\usepackage{subcaption}
\usepackage{colortbl}
\usepackage{tikz}
\usepackage{algorithm}
% \usepackage{algorithmic}
\usepackage{algpseudocode}
\usepackage{xparse}
\usetikzlibrary{positioning}


% \usepackage[utf8]{inputenc} % allow utf-8 input
\usepackage[T1]{fontenc}    % use 8-bit T1 fonts
% \usepackage{url}            % simple URL typesetting
\usepackage{booktabs}       % professional-quality tables
% \usepackage{amsfonts}       % blackboard math symbols
% \usepackage{nicefrac}       % compact symbols for 1/2, etc.
% \usepackage{microtype}      % microtypography
% \usepackage{}
\usepackage{multirow}
% \usepackage{pbox}
% \usepackage{enumitem}
% \usepackage{bbm}
% \usepackage{unicode-math}
% % \usepackage[table,xcdraw]{xcolor}
% \usepackage{xfrac}
% \usepackage{pifont}% http://ctan.org/pkg/pifont
% \usepackage{dblfloatfix}

% % For figures
% \graphicspath{{figs/}}

\usepackage[prependcaption,textsize=tiny]{todonotes}



\DeclareCaptionLabelFormat{andtable}{#1~#2  \&  \tablename~\thetable}
\DeclareDocumentCommand{\particles}{ O{} O{} O{}}{^{#2}X_{#1}^{#3}}
% \particle[time][index][section]
\DeclareDocumentCommand{\particle}{ O{} O{} O{}}{^{#2}x_{#1}^{#3}}
\newcommand{\state}{x}

\newcommand{\sampled}{\tilde}
\newcommand{\subcap}[1]{\textbf{(\subref{#1})}}
\newcommand{\maction}{\mathcal{M}}
%% \newcommand{\particles}{\mathcal{P}}
%% \newcommand{\particle}{x}
\newcommand{\bin}{b}
\newcommand{\groups}{L}
\newcommand{\measurement}{m}
\newcommand{\measurementSet}{M}
\newcommand{\totalWeight}{\mathcal{W}}
\newcommand{\feature}{\mathcal{S}}
\renewcommand{\algorithmicrequire}{\textbf{Input:}}
\renewcommand{\algorithmicensure}{\textbf{Output:}}
\def\BState{\State\hskip-\ALG@thistlm}

\algnewcommand\algorithmicforeach{\textbf{for each}}
\algdef{S}[FOR]{ForEach}[1]{\algorithmicforeach\ #1\ \algorithmicdo}


% Approximately 1" margins, more space on binding side
%\usepackage[letterpaper,twoside,vscale=.8,hscale=.75,nomarginpar]{geometry}
%for general printing (not binding)
\usepackage[letterpaper,twoside,vscale=.8,hscale=.75,nomarginpar,hmarginratio=1:1]{geometry}

% Provides a draft mark at the top of the document. 
% \draftstamp{\today}{DRAFT}

\begin {document} 
\frontmatter

%initialize page style, so contents come out right (see bot) -mjz
\pagestyle{empty}

\title{ %% {\it \huge Thesis Proposal}\\
{\bf Some fancy title}}
\author{Guan-Horng Liu}
\date{\today}
\Year{2017}
\trnumber{}

\committee{
George A. Kantor \\
Manuela Veloso \\
Devin Schwab 
}

\support{}
\disclaimer{}

% copyright notice generated automatically from Year and author.
% permission added if \permission{} given.

%% \keywords{}

\maketitle

%% \begin{dedication}

%% \end{dedication}

\pagestyle{plain} % for toc, was empty

%% Obviously, it's probably a good idea to break the various sections of your thesis
%% into different files and input them into this file...

\begin{abstract}

% Sensor fusion is indispensable to improve accuracy and robustness in an autonomous navigation setting. However, in the space of end-to-end sensorimotor control, this multimodal outlook has received limited attention. In this work, we propose a novel stochastic regularization technique, called \textit{Sensor Dropout}, to robustify multimodal sensor policy learning outcomes. We also introduce an auxiliary loss on policy network along with the standard DRL loss that help reduce the action variations of the multimodal sensor policy. Through empirical testing we demonstrate that our proposed policy can 1) operate with minimal performance drop in noisy environments, 2) remain functional even in the face of a sensor subset failure. Finally, through the visualization of gradients, we show that the learned policies are conditioned on the same latent input distribution despite having multiple sensory observations spaces - a hallmark of true sensor-fusion. This efficacy of a multimodal policy is shown through simulations on TORCS, a popular open-source racing car game. 
% Suitable metrics have been devised to study this behavior and highlight its applicability to other domains that operate in multimodal settings. 


  TODO
\end{abstract}

\begin{acknowledgments}

The author would like to thank ...
This thesis and my work in graduate school would not be possible without the co

my advisor, George Kantor, for both and financial support

Thanks all of the members in the Yamaha project 
especially Jay and 
for supporting my on-field testing experiment


Humphrey
Hsiao Ming, 

Lab mates, Po-Wei Chou, Wei-Hsin Chou, Daniel Lu

my parents and Bonni Chen

\end{acknowledgments}



\tableofcontents
\listoffigures
\listoftables

\mainmatter

%% Double space document for easy review:
%\renewcommand{\baselinestretch}{1.66}\normalsize

% The other requirements Catherine has:
%
%  - avoid large margins.  She wants the thesis to use fewer pages, 
%    especially if it requires colour printing.
%
%  - The thesis should be formatted for double-sided printing.  This
%    means that all chapters, acknowledgements, table of contents, etc.
%    should start on odd numbered (right facing) pages.
%
%  - You need to use the department standard tech report title page.  I
%    have tried to ensure that the title page here conforms to this
%    standard.
%
%  - Use a nice serif font, such as Times Roman.  Sans serif looks bad.
%
% Other than that, just make it look good...


\chapter{Introduction}
\subfile{Introduction/intro.tex}

\chapter{Related Work} \label{chap:related_work}
\subfile{RelatedWork/relatedWork.tex}

\chapter{Model Predictive Planning} \label{chap:rrtplanner}
\subfile{RRTPlanner/rrtplanner.tex}

\chapter{Learning End-to-end Multimodal Sensor Policy} \label{chap:multimodalDRL}
\subfile{MultimodalDRL/multimodalDRL.tex}

\chapter{Learning from Demonstration using DIRL} \label{chap:dirl}
\subfile{DIRL/dirl.tex}

\chapter{Conclusion and Future Work} \label{chap:conclusion}
\subfile{Conclusion/conclusion.tex}

%\appendix
%\include{appendix}

\backmatter

%\renewcommand{\baselinestretch}{1.0}\normalsize

% By default \bibsection is \chapter*, but we really want this to show
% up in the table of contents and pdf bookmarks.
\renewcommand{\bibsection}{\chapter{\bibname}}
%\newcommand{\bibpreamble}{This text goes between the ``Bibliography''
%  header and the actual list of references}
% \bibliographystyle{srtnat}
\bibliographystyle{unsrtnat}
\bibliography{MDRLBib,RRTPlannerBib,DIRLBib} %your bib file
\nocite{*}

\end{document}
